\newpage
\chapter{Evaluating Project}

When designing and implementing a web-application there are several measures one must consider to evaluate the success of the project. The web-application is meant to be used by a wider range of users. Therefore, performance is not the only aspect of the application I must evaluate. User experience is equally important. 

In order to evaluate the user experience and accessibility of the application the web-application should become available to pilot users early on. Through the feedback received, issues can be identified and the experience can be improved accordingly. Use of the web-application requires certain knowledge of argumentation. By creating a simple informative section on the website of the web-application the concept should be introduced to a level that would allow users to provide feedback on their interaction with the application.

Performance wise the system will be tested with a series of example frameworks, to ensure its performance both in validity and in computational speed. The existing systems do compute with a satisfactory speed. The web-application should not hinder this performance significantly. To ensure that the application performs well, the test cases will be vary in terms of complexity and size.

Lastly, certain automated tests will be used during the development phase to ensure that the system is robust against certain likely software issues.  One of the aims of the project is for the application to be extendible in the future. The existence of these automated tests has the additional benefit of making the application more extendible by other developers in the future. By providing these tests, we allow the developers to ensure that the initial functionality of the application is not obstructed by any additional functionality they might introduce. Such, tools are important in ensuring that the application is extendible beyond the scope of this project.
