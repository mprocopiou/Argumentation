\newpage

\chapter{Conclusions}

The web application has been developed to an extend that allows it to perform according to the requirements specified at the beginning of the project. It provides a portal for users to experiment with ABA frameworks using an array of derivation engines. However, due to the nature of the project it was built with possibility of further extending it in the future. It is important to look into the potential this system generates, how it can be extended and what are its contributions to the world of ABA and Computing in general.

\section{What comes next?}
Extendibility and flexibility have always been at the heart of this project, throughout both the design and implementation stages. The potential of the current web application lies in the ability to use the functionality as a middle-ware on top of which further application can be developed. Additionally, further derivation engines can be adapted so as to be included in the back end of the project. This can allow for comparison to be made between engines, but also can enable the inclusion of derivation engines with specific purposes.

\subsection{Building on top of the Web---Application.}
Due to the modular approach followed when implementing the web application and due to the definition of input commands that can be used to interact with the system, extra applications can be built on top using our web application as middle-ware. The system can be considered as in Figure [TODO], which shows how the system allows a black---box approach to the derivation engines. An input can be provided, it will be processed and the relevant output will be given to the user. The specifics have already been elaborated on during the report, however what should be mentioned is the systems ability to accommodate extensions on both the input mechanisms and the output mechanism. 

Reconsidering figure [TODO] we can see that extensions can be built on top of the application both on the input and the output end. These extensions can enable alternative inputs or alternative methods of interpreting the output. As long as the extensions are built so as to use the input language provided by the application and understand the format of the output, such extensions can be integrated seamlessly. 

As an example consider Fan and Tonis suggestion [TODO] ADD REFERENCE, by which an Abstract Argumentation framework (AA) can be directly mapped to an ABA framework. A possible input extension would allow the users to input an AA framework which would then be mapped to the respective ABA framework. This can then be automatically formed in the input language of our application and be processed for a derivation. This would extend our application to allow derivations for AA framework to be carried out as well. Similarly other extensions can be implemented that could provide additional functionality, especially when it comes to building applications around real---life problems.


\subsection{Real World Application.}
The strength of ABA lies in its ability to not only provide a true or false argument, but also provide a valid derivation of why that was the outcome. This can be exceptionally useful in decision making problems or providing advisory tools. Applications of argumentation in general can be found in the area of medicine and law, where they provide tools that assist in establishing the validity of claim. For example, provided the correct ABA framework an application can be used to derive whether a diagnosis provided by a doctor is possible according to the framework.

Applications can be built on top of the existing one that are designed specifically for such a real---world scenario. Essentially our web application can be used as a back---end for a more specific application. For ABA to be truly appreciated by the public such tangible, real---life implementations will have to be built. Tools can be built on top of the existing web application in such a way that allows for a useful service to be provided, while abstracting the user from the specialised knowledge of ABA frameworks.

As a suggestion it is worth mentioning the possibility that an advice---providing tool that could possibly work based on ABA could be a tool that provides financial feedback to Small and Medium Enterprises (SMEs) or to people's personal finance. Rules and assumptions can be derived from financial statements that could provide a better understanding of what the users could do to improve their financial situation. This is closely tied to the study of Managerial Accounting were certain calculations and variances act as indicators of a company's strengths and weaknesses. Such an application could take in the relevant financial statements and then return a report concerning suggestions for improvement based on several derivations handled in the back---end.

\section{Concluding remarks.}
In conclusion the web application can be considered as more than just a derivation tool. In fact it can be considered as a platform on which other tools can be implemented with further development. It can also be used as a portal for accessing and comparing various derivation engines. We provided a unified input mechanism that is abstracted by the specifics of the derivation engines, allowing the users to ignore the specifics of what goes on behind. We also extended the existing derivation engines by providing a grounder for the frameworks specified. By handling this as a pre---processing stage we can ensure that the derivation engine will always receive a grounded framework, which it is able to handle it. Additionally we further extended the derivation engines by implementing ideal based semantics derivations, thus extending the engines functionality further. All of this has been wrapped into an easy to use web application that users can access and experiment with ABA framework.

In its current implementation the system would best serve as an educational tool for introducing the users to ABA frameworks. The simple input required and the clean visualisation provided enables users to easily understand the derivation of the target they specified. The ability to do so for 3 different semantics allows them to learn about the differences between the different semantics. However having built the system to provide a middle---ware solution for other applications and engines the project has great potential for further development. Additional engines can be added and further applications can be built both on top of the input and the output of our solution. This way, eventually, the web application can grow into a one--stop---shop for all things ABA.
